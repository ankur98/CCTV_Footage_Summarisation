\chapter{Introduction to CCTVVS}

The number of \gls{cctv} surveillance cameras is increasing everyday leading to
a huge amount digital video information being captured and stored. Millions of
CCTV cameras run 24 hours a day, sometimes even streaming the content over the
internet for people to monitor. This data is however in a raw, unprocessed
format.  In most cases, video with little to no motion is being captured,
which wastes lot of storage. The process of watching or analysing the footage is
also time consuming and laborious.

This project proposes to build a CCTV video summariser to eliminate this
redundancy in stored data and give the user a short summary consisting of
important information that is relevant for the specific use case. The system
proposes to incorporate a query based summary generation to generate more
relevant summaries. Users can choose the required objects and events to generate
a short yet precise summary with only relevant information, saving more time.

The process takes place over two phases, the real-time and query phase. The
real-time phase reads the CCTV footage, identifies clips of interest, extracts
the “flow-tubes”, and stores them after tagging them with the respective object
tags. A query phase would then involve the user selecting the required tags and
objects of interest, which are chosen. The tubes are rearranged using simulated
annealing algorithm, before being blended into a generated time-lapsed
background using an optimal technique like poisson blending.

This method reduces the manual work of going through hours of footage
looking for relevant events by automatically creating a summary. It can be
mainly used for security purposes, by the police forces for detection of crimes
and suspicious activities.


\section{State of the art developments}

In this section, existing methods for summarizing videos and related
information that helps achieve this task are discussed.

In~\cite{rav2006making}, Yael Pritch and Alex Rav-Acha propose a method to
effectively generate a synopsis of an endless video stream that can also be
used as an index into the main video. An online phase includes tube detection
in spatio-temporal domain, insertion of these tubes into an object queue, and
removal upon reaching a space limit. The response phase then constructs a
time-lapse video of the changing background, selection and stitching of tubes
into a coherent video. Min-cut algorithm along with background subtraction has
been used for extracting moving objects. Activity, collision and temporal
consistency costs have been used a parameters for optimal tube arrangement.

In~\cite{pritch2008nonchronological}, Shmuel Peleg and Yael Pritch, have
presented a dynamic video synopsis technique where most of the activity in the
video is condensed by simultaneously showing several actions, even when they
originally occurred at different times.

In~\cite{rodriguez2010cram}, (CRAM: Compact Representation of Action in Movies),
Mikel Rodriguez generates a compact video representation of a long sequence,
which while preserving the general dynamics of the video features only the
essential components. From the given input video, optical flows are generated.
These are then represented as vectors in Clifford Fourier domain. Dynamic
regions of flow are then identified within the phase spectrum volume. The
likelihood of activities of relevance are then computed by correlating it with
spatio-temporal maximum average correlation height filters. The final summary
is then generated by a temporal-shift optimization. Although this method could
detect specific actions, it couldn’t keep all the events in the final summary.

Sarit Ratovitch, Avishai Hendel and Shmuel Peleg, in their paper titled
Clustered Synopsis of Surveillance Video~\cite{pritch2009clustered}, present a
different approach to generating video summaries, based on clustering of
similar activities. Objects with similar activities are easy to watch
simultaneously, which also makes spotting of outliers easier. This method is
also suitable for creation of ground truth data. This paper covered three main
topics, the definition of distance between activities, clustering of similar
activities and efficient presentation of video summaries using obtained
clusters.

In~\cite{porter2003shortest}, (A Shortest Path Representation for Video
Summarization), a new approach for video summarization is presented to select
multiple key frames within an isolated video shot where there is camera motion,
causing significant scene change. This can be done by determining dominant
motion between frame pairs whose similarities are represented using a directed
graph. A* algorithm is used to detect the shortest path and designate key
frames. The overall set of key frames depict the essential video content and
camera motions.

~\cite{zivkovic2004improved} presents a very successful and highly used method
for adaptive pixel-level background subtraction. Each pixel has probability
density function separately. A pixel is considered to be part of the background
if its new value if well described by its density function. This paper was an
improvement on previous models which used Gaussian mixture models with
efficient update equations.

In~\cite{redmon2016you}, an extremely fast object detection model, the
\gls{yolo} model is described. While prior object detectors used classifiers to
detect, this paper proposes object detection as a regression problem to
spatially separated bounding boxes and associated class probabilities. A single
neural network is used to predict both bounding boxes and its class probability,
making end-to-end optimization easy. Although YOLO makes more localization
errors, it is less likely to predict false positives compared to other object
detectors like \gls{ssd}, \gls{rcnn} and Faster RCNN.

In~\cite{smith1998video}, a combination of camera motion, object motion and
text detection are used to identify which scenes are important in the input, and
hence should be maintained in the summary.

\section{Motivation}

The prime motivation was the fact that manual analysis of CCTV footage is time
consuming and laborious. With the increase in the number of CCTV’s being used,
it is necessary to have a smart method to generate a summary of footage which
highlights what is necessary and removes the rest. The idea of this system was
to help the users save tons of time and manual labour.

While there are basic methods like fast forwarding, adaptive fast-forwarding~
\cite{petrovic2005adaptive}, and motion detection, the aim is to improve the
efficiency further by generating a shorter and more relevant summary. This
project also helped us to explore deeper into the domain of computer vision,
towards which we share a common interest.

\section{Problem Statement}

Manual analysis of CCTV footage is extremely time consuming and laborious. Thus
there is a need for smart tools to generate useful insights and summaries.

The task at hand is to develop an efficient method to generate a summary from a
given input CCTV footage. It must be able to preserve important content while
eliminating unnecessary details, thus saving hours of manual work hours spent
going through the entire video. In addition, there is a lack of a smart
query-based system for summarization. The system must generate a summary based
on user query.

\section{Objective}

The objectives of the system are set with the requirements of each module.
\begin{itemize}
    \item To develop a video summarization technique, that will take
    efficiently handle overlapping of events in the video.
    \item To add the functionality of generating a summary based on user query.
    \item To build a suitable UI and integrate with the
    available back-end.
\end{itemize}

\section{Scope}

The main area of application of this project is for security purposes. It can be
used for detection of crimes, detecting suspicious activities
\cite{boiman2007detecting} etc. Hence it will be of use to the security
forces and the police.

\section{Methodology}

The project has been split into 2 phases – the real-time phase and query phase.

Real-time phase includes all those processes that work on the input data from
the CCTV before storing the intermediate result in the database. These include
background masking, to separate the foreground and background, motion detection
to capture frames of interest, tube extraction to store these frames of
interest, object detection to gather metadata about the events in the video.
This phase also includes generating a static background of the input video.
The generated data is then stored in the database.

The next phase, the query phase, then takes in user query, selects the tubes
based on the query, rearranges the tubes using Simulated Annealing algorithm,
and then blends these rearranged tubes into the background using Poisson
Blending to generate the video summary.


\section{Organization of Report}

This section gives the overall picture of the many chapters in this report.

\begin{itemize}
    \item \textbf{Chapter 2} gives an overview of the project domain which
    describes the details of the software and techniques used to carried out
    the project.

    \item \textbf{Chapter 3} is on Software Requirement Specification which
    describes the assumptions and dependencies, user characteristics,
    functional requirements and constraints of the project.

    \item \textbf{Chapter 4} is High Level Design which elucidates the design
    phase in Software Development Life Cycle. This chapter talks about the
    design considerations like architectural strategies, general constraints,
    development methods and. It explains the project System Architecture and
    Data Flow Diagrams.

    \item \textbf{Chapter 5} is Detailed Design which explicates the two
    project modules. The functionality of each module represented as a
    flowchart is explained in this section.

    \item \textbf{Chapter 6} is Implementation which describes the technology
    used in the system. This section also explains programming language,
    development environment, code conventions followed.

    \item \textbf{Chapter 7} is on Software Testing which elaborates the test
    environment and briefly explains the test cases which were tried out during
    unit, integration and system testing.

    \item \textbf{Chapter 8} is Experimental Results which mentions the results
    found by the experimental analysis on different data sets. It talks about
    the inference made from the results.

    \item \textbf{Chapter 9} is Conclusion conveying the summary, limitations
    and future enhancements of the project.
\end{itemize}


\section{Summary}

This chapter deals with the introduction to the topic. The existing systems for
video summarization are discussed. Research work, study material and papers on
accessory software is also discussed. It also discusses in detail about the
motivation, problem statement and the objectives of the project.