\chapter{Software Requirement Specification of CCTVVS}

Software Requirements Specification is detail description of system behavior
that is constructed. It integrates the functional and nonfunctional requirements
for software to be constructed. The functional requirements describe what
exactly the software must do and the non-functional requirement includes the
constraint on the design or implementation of the system. A function is
described as a set of inputs, the behavior, and outputs. A non-functional
requirement is a requirement that specifies criteria that can be used to judge
the operation of a system, rather than specific behaviors.

\section{Overall Description}

This section describes the general factors which affects system and the
requirements. The software developed should provide means to configure order
assignment, generate master files with required restrictions and must handle
concurrent assignment of orders to reconciliation specialists. This section
also deals with user characteristics, constraints on using the system and
dependencies of the system on other applications.

    \subsection{Product Perspective}
    The system should be versatile and easy to use. It should be flexible and
    the response time should be quick. The system is composed of several modules
    performing different tasks and they must be well co-ordinated. The system
    developed should be easy to deploy and maintain. The intended end users for
    this are mostly CCTV operators and security managers. The system must be
    self-explanatory as it will be used by laymen or people with minimum
    technical experience.

    \subsection{Product Functions}
    The sole purpose of the product is to generate semantic, concise and
    relevant summaries for the specified time periods. The product must be
    versatile enough to handle sophisticated queries which considers several
    parameters like type of subjects, time period, summary length, degree of
    importance of events, duration of events to be considered etc.

    \subsection{User Characteristics}
    The end users of the system are mostly of to two roles, CCTV operators and
    security managers. Since they are not technical specialists, the user
    interface is expected to be simple and easy to use. It can be assumed that
    the general user of this application is versed in form-based inputs and has
    a grasp on basic English to use the interface accordingly. It is also
    assumed that the user can make all kinds of queries, both valid and invalid,
    and it is up to to the software to validate the queries and provide
    responsive feedback accordingly.

    \subsection{Constraints and Dependencies}
    The dependencies and constraints that the project has are:
    \begin{itemize}
        \item The first major constraint is the memory constraint. The number
        of flow-tubes it can hold in memory is directly proportional to the
        amount of events which can be optimized at a time.
        \item The system is highly dependent on having accurately tuned
        hyper-parameters, whether it be the threshold for motion detection or
        the learning (exploration/exploitation) rate used in the optimization
        technique, must be tuned specifically for each use-case.
        \item The number of categories it can recognize in a video is
        constrained on the diverseness of the dataset used to train the
        deep learning model used.
        \item A major dependency of the project is the \gls{openCV} parallel
        implementation of several advanced algorithms like MOG and Poisson
        Blending.
    \end{itemize}

\section{Specific Requirements}

This section covers all software requirements with sufficient details to be made
use of by the designers to build a system to satisfy requirements. The product
perspective and user characteristics do not state the actual requirements needed
that the system but rather state how the product should work with respect to
user convenience. The specific requirements are actual data with which the
customer and software provider can agree. The final system is expected to
satisfy all the requirements mentioned here.

    \subsection{Functional Requirements}
    The functional requirements of the system are:
    \begin{itemize}
        \item The system must automatically validate user queries before
        processing them
        \item The system must produce video summaries based on the user query
        given
        \item The system must label each event shown in summary with relevant
        time stamps
        \item The system must be able to identify and tag events in real-time
        \item The system must be able to merge and rearrange events without
        losing meaning
        \item The system must be able to seamlessly blend subjects with
        generated time-lapsed background
        \item The system must be able to generate a relevant background based on
        the density of events distributed across the queried period of time
        \item The system must be able to generate summaries not longer than ten
        percent of the queried period length
        \item The user query form must be versatile enough to encompass all
        forms of relevant queries
        \item The system must have a feature to manually tune the various
        possible hyper-parameters from the user interface
    \end{itemize}

    \subsection{Performance Requirements}
    Performance requirements for the system include the following:
    \begin{itemize}
        \item The system should be able to transmit the video feed with minimal
        loss of packets
        \item The delay with respect to the video transmission should be minimal
        \item The control signal should be transmitted at real time
        \item The obstacle detection should happen in real time
    \end{itemize}

    \subsection{Supportability}
    The UI will be web-based and hence can be used on any system. While the
    system is built on python and open-source tools which are ubiquitous in
    modern day PCs, however it is recommended to use computers with large amount
    of RAM for fast performance, as the algorithm developed for optimization
    requires the system to hold significantly large arrays in memory.

    \subsection{Software Requirements}
    The different software requirements required by the application are as
    follows:
    \begin{itemize}
        \item Operating System:
        \begin{itemize}
            \item OS X Yosemite and higher versions
            \item Windows 7 and higher versions
            \item Linux 16.04 LTE and higher
        \end{itemize}
        \item Platform: Python Virtual Environment
        \item Language: Python
        \item Interface: Python GUI
        \item IDE/tool:
        \begin{itemize}
            \item Visual Studio Code
            \item Jupyter Notebook
        \end{itemize}
    \end{itemize}

    \subsection{Hardware Requirements}
    The various hardware requirements of the system are:

    \begin{itemize}
        \item Platform/CPU
        \begin{itemize}
            \item Modern multi-core processor (Ex: Intel Core i7 6700HQ) on
            x86/ARM platform
        \end{itemize}
        \item Memory
        \begin{itemize}
            \item Minimum: 8GB
            \item Recommended: 16GB or higher
        \end{itemize}
        \item Storage
        \begin{itemize}
            \item 64GB or higher for storing an entire day of video (depending
            on video resolution)
        \end{itemize}
    \end{itemize}

    \subsection{Design Constraints}
    The system is designed to be versatile and flexible. The design constraints
    include a user friendly interface. It should be able to handle requests in
    short time periods. The system must ensure that database is always in a
    consistent state. Also, the system must be built to handle all possible
    inputs so as to make the feature comprehensive enough for non-technical
    end users. The system must be able to validate input queries.

    \subsection{Interfaces}
    This section describes the interfaces in detail. The two types of
    interfaces involved are User Interfaces and Software Interfaces.
        \subsubsection{User Interfaces of the system}
        There is a single user interface from which the software accept
        parameters from the user and downloads the summary created. It must be
        form based and have the following parameters:
        \begin{itemize}
            \item Input source \& output file path
            \item Time period
            \item Summary length
            \item List of tags of subjects to include
            \item Learning rate
            \item Threshold for size of events
        \end{itemize}

        \subsubsection{Software Interfaces of the system}
        The following software interfaces are implemented in the application:
        \begin{itemize}
            \item The different modules developed in the system uses standard
            pre-defined data structures as interfaces between each other.
            \item Each module is developed as objects hence, the data
            internally is exposed by helper functions.
        \end{itemize}


    \subsection{Non-Functional Requirements}
    These are the requirements that specify criteria that can be used to judge
    or evaluate the operation of the system rather than specific behavior and
    are not directly concerned with the specific functions delivered by the system.

    \begin{itemize}
        \item Efficiency: The system shall perform at best possible efficiency
        in all internal operations.
        \item Availability: The system must be up and running at all times and
        must accommodate users
        \item Security: The system must be well built with no security threats.
        \item Uniformity: The system must perform on any standard browser and
        any OS in the same way, while producing fairly similar summaries each
        time a similar query is made.
        \item Speed: The system must have a short latency period and must be
        responsive
    \end{itemize}

\section{Summary}

The specific requirements and constraints that must be kept in mind while
building the application have been detailed in this chapter. These include the
hardware requirements, software requirements and functional requirements for
automation of reconciliation. Also, this chapter cites the various assumptions
being made by the developer of the system for auto assignment of orders.  All
these have to be managed while building and running the system.